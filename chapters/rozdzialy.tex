
\chapter{Część zasadnicza}

\section{Studium przypadku}
% Opis jednego procesu coachingowego z zachowaniem poufności poprzez zapewnienie anonimowości osoby coachowanej (min. 5 sesji z jednym klientem.

W tej sekcji został opisany proces coachingowy realizowany w trakcie trwania studiów podyplomowych, prowadzony przez autora pracy.

\subsection{Początkowe wyzwania i cele klienta}
% punkt wyjścia - wyzwania i cele klienta

\begin{itemize}
  \item Pierwsze spotkanie miało na celu wybranie obszaru i celów do pracy nad. Klientka wybrała obszar zawodowy, gdzie po niedawno otrzymanym awansie stoją przed nią nowe wyzwania.
  \item Szczegółowe cele dotyczyły: poprawienia metod organizacji pracy członków koordynowanego zespołu, skuteczne stawianie wymagań, a także poprawienie metod przekazywania informacji zwrotnej.
  \item Dla każdego z celów klientka chciała określić mierzalne oznaki realizacji. Udało się je określić dla:..., nie udało się określić ich dla: ... .
\end{itemize}

\subsection{Proces coachingowy}
%) proces - nad czym pracowano na jakich poziomach zachodziły zmiany co stanowiło szególne wyzwanie w pracy akurat z tym klientem?

\subsection{Uzyskane efekty i dalsze cele}
%) efekty - stan docelowoy - co zrealizowano, jakie dalsze postępowanie byłoby wskazane?

\subsection{Retrospekcja z punktu widzenia coacha}
%) feedback - czego, jako coach w tym procesie dowiedziałem się o sobie/ swoich kompetencjach / postawie coacha - jakie dlasze zadania rozwojowe w roli coacha przede mna?}

\begin{itemize}
  \item Nikła wiedza na temat coachingu mimu kontaktu z wykształconą, zaradną osobą. Obawy wynikające z braku wiedzy
  \item Uzyskaliśmy dużo lepsze efekty niż przy pracy względem których istnieje wysokie zaangażowanie emocjonalne
  \item Wniosek klienta: sama też bym to potrafiła, ale dzięki temu naprawdę to robię
\end{itemize}

\section{Własne doświaczenie coachingu}

\TODO{na podstawie sesji kiedy bylem osoba coachowana i cwiczen, ktore byly zwiazane z autocoachingiem. Jakie korzysci proces coachingowy przyniosl mi, na jakich poziomach piramidy Diltsa obserwuje zachodzace zmiany? Co ma dla mnie najwieksze znaczenie/ jest dla mnie najwieksza wartoscia z zaangazowania sie we wlasny proces coachingowy?}

\section{Zakres zastosowań coachingu}

\TODO{Jak teraz rozumiem zakres zastosowan coachingu? Ktore sposrod standardow etycznych ICF oraz Izby Coachingu uwazam za szczegolnie istotne z wlasnej praktyki zawodowej - uzasadnij z jakich powodow? Komu, kiedy i w jakich sytuacjach zaoferowalabym coaching? w jakich przypadkach przekazalbym klientowi kontakt do innego (jakiego?) specjalisy?}
