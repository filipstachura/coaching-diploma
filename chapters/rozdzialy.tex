
\chapter{Część zasadnicza}

\section{Studium przypadku}
\TODO{Opis jednego procesu coachingowego z zachowaniem poufności poprzez zapewnienie anonimowości osoby coachowanej (min. 5 sesji z jednym klientem).
a) punkt wyjścia - wyzwania i cele klienta
b) proces - nad czym pracowano na jakich poziomach zachodziły zmiany co stanowiło szególne wyzwanie w pracy akurat z tym klientem?
c) efekty - stan docelowoy - co zrealizowano, jakie dalsze postępowanie byłoby wskazane?
d) feedback - czego, jako coach w tym procesie dowiedziałem się o sobie/ swoich kompetencjach / postawie coacha - jakie dlasze zadania rozwojowe w roli coacha przee mna?}

\section{Własne doświaczenie coachingu}

\TODO{na podstawie sesji kiedy bylem osoba coachowana i cwiczen, ktore byly zwiazane z autocoachingiem. Jakie korzysci proces coachingowy przyniosl mi, na jakich poziomach piramidy Diltsa obserwuje zachodzace zmiany? Co ma dla mnie najwieksze znaczenie/ jest dla mnie najwieksza wartoscia z zaangazowania sie we wlasny proces coachingowy?}

\section{Zakres zastosowań coachingu}

\TODO{Jak teraz rozumiem zakres zastosowan coachingu? Ktore sposrod standardow etycznych ICF oraz Izby Coachingu uwazam za szczegolnie istotne z wlasnej praktyki zawodowej - uzasadnij z jakich powodow? Komu, kiedy i w jakich sytuacjach zaoferowalabym coaching? w jakich przypadkach przekazalbym klientowi kontakt do innego (jakiego?) specjalisy?}
