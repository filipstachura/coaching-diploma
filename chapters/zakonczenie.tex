\chapter{Zakończenie}

% Jak moge wykorzystac umiejetnosci i doswiadczenia nabyte w trakcie studiow i kontynuowac swoj rozwoj w roli coacha?
% Jakie dzialania zamierzam podjac, kto bedzie stanowic moja docelowa nisze?
% (z jakimi klientami i w jakich kontekstach mam zamiar pracowac?)
% Jaki zakres specjalizacji chcę rozwijac? Jakie dzialania praktyczne i kroki dalszego rozwoju (obszary doskonalenia) zamierzam podjac?

Autor pracy zamierza kontynuować swój rozwój osobisty i wykorzystać zdobyte w trakcie trwania studiów umiejętności na polu zawodowym.
W obecnym momencie nie zamierza, ani nie planuje w najbliższej przyszłości pracować jako coach, ponieważ prowadzi spółkę związaną z działalnością
zupełnie innego rodzaju. Rozwój osobisty traktuje jako pasję i na tym polu widzi swój potencjał do rozwoju, a w dalekiej przyszłości
chciałby realizować się w tej pasji pomagając innym.

Mając to na uwadze autor chciałby pielęgnować zdobyte umiejętności poprzez aktywne wykorzystywanie ich do autocoachingu i w życiu zawodowym.
W swojej dalekiej przyszłości chciałby pracować jako coach lub mentor dla osób lub zespołów realizujących projekty obarczone dużym ryzykiem.
Dopuszcza więc również możliwość pogłębienia swoich umiejętności w obszarze mentoringu lub szkoleń w niedługiej przyszłości.
Jako pierwszy cel na drodze realizacji tej wizji autor planuje udział w imprezach w obszarze nowych technologii
w charakterze coacha lub mentora - w perspektywie kilku kolejnych lat. Na tego typu wydarzeniach interdyscyplinarne zespoły mają za zadanie
w okresie kilku dni wymyślać innowacyjne rozwiązania wraz z pomysłem na ich komercjalizację.

Dodatkowo autor sam planuje rozpocząć pewien czas po zakończeniu studiów kolejny proces coachingowy - tym razem w charakterze osoby coachowanej.
