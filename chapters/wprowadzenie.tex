\chapter{Wstęp}

% Sgormułowanie własnej, indywidualnej definicji coachingu (uszczegółowioną i uzasadnioną). W jaki sposób osobiście
% rozumiem czym jest Coaching oraz kim /jestem/staję się w roli coacha?

Niniejsza praca poświęcona jest doświadczeniom zebranym przez autora w trakcie studiów podyplomowych na kierunku
\emph{Coaching profesjonalny}. Omawiane są również obserwacje i wnioski bazujące na tych doświadczeniach.
Praca została podzielona na trzy części. We wprowadzeniu została sformułowana indywidualna definicja coachingu,
która następnie została uzasadniona i porównana z definicją \emph{ICF}. Kolejna część, stanowiącą trzon pracy,
została podzielona na trzy podrozdziały, zawierające odpowiednio:
\begin{enumerate}
  \item Studium przypadku pracy z klientem, w którym dokładnie opisany jest jeden pełen proces coachingowy  , poprowadzony
      przez autora pracy w trakcie trwania studiów.
  \item Omówienie przez autora własnych doświadczeń związanych z coachingiem z perspektywy klienta. Autor bazuje na
      ćwiczeniach realizowanych z innymi studentami w ramach studiów i autocoachingu.
  \item Opinie na temat obszaru zastosowań procesu coachingowego oraz standardów etycznych.
\end{enumerate}
W ostatniej części pracy autor przedstawia swoją wizję dalszego rozwoju osobistego, ze szczególnym uwzględnieniem
obszaru umiejętności coachingowych. \\

\section{Indywidualna definicja coachingu}
Coaching jest forma wspomagania rozwoju, opartą o dialog pomiędzy coachem a klientem. Coach jest osobą odpowiedzialną za prowadzenie procesu.
Ze względu na swój niedyrektywny charakter, a również często niemierzalność efektów procesu coachingowego forma ta doczekała się
licznych definicji. Również w trakcie trwania studiów podyplomowych studenci mogli spotkać się z wieloma różnymi definicjami,
zależnie od wykładowcy prowadzącego zajęcia. Każda z tych definicji na swój sposób określa warunki tego procesu.
Jednocześnie w zdecydowanej większości przypadków definicje te nie są względem siebie sprzeczne, a raczej uzupełniają się
wzajemnie, tworząc razem pełniejszy obraz i budując intuicję słuchaczy o tym czym \emph{coaching} jest.

Pomimo wymienionych trudności stojących na przeszkodzie jednoznacznego zdefiniowania tej formy rozwoju istnieje grupa wymagań wspólnych dla
wszystkich definicji. Będą to na przykład te wymagania, które mówią o tym, że coaching to proces lub że jest współpracą pomiędzy
coachem a klientem. Indywidualna definicja autora tej pracy ma następującą treść:
\begin{defn}
  \emph{Coaching} jest dobrowolnym i niedyrektywnym, ograniczonym czasowo procesem partnerskiej współpracy,
  ukierunkowanym na rozwój klienta poprzez skupienie się na teraźniejszości i przyszłości.
  \label{definicja}
\end{defn}

W celu dokładnego wytłumaczenia przedstawionej definicji poszczególnie jej części zostały omówione niezależnie poniżej:

\subsection{Coaching jako proces dobrowolny}
Jednym z podstawowych założeń procesu coachingowego jest jego swobodna i nieobowiązkowa forma. Dotyczy to zarówno klienta, ale jest
również istotne z punktu widzenia coacha. Ponieważ obie strony mają świadomość swojego dobrowolnego uczestnictwa w spotkaniach pozwala to budować
zdrową relację opartą o niezależność wszystkich zaangażowanych osób.

\subsection{Coaching jako proces niedyrektywny}
Fundamentalnym założeniem jest powierzenie klientowi pełnej odpowiedzialności za podejmowane przez niego decyzje. W sposób naturalny
warunek ten jest spełniony przez brak dyrektywnych rad i podpowiedzi od coacha. Ponieważ coach zadaje wyłącznie pytania klientowi,
a klient sam na podstawie pytań dochodzi do wniosków i rozwiązań, a ostatecznie swoich decyzji to naturalne jest branie odpowiedzialności
za te decyzję przez klienta. Zaakceptowanie tego wymagania powinno być warunkiem niezbędnym każdego procesu coachingowego.

Dodatkowym, bardzo ważnym czynnikiem jest również wysoka uwaga coacha na wystrzeganie się dyrektywności "\emph{nie wprost}". O takiej formie
dyrektywności mówi się wtedy, gdy coach przez zaniedbanie lub świadome działanie nakłania klienta do podjęcia pewnej decyzji bez uczciwego przeanalizowania
innych możliwości. \\

\subsection{Coaching jako proces ograniczony czasowo}
Zgodnie z tym stwierdzeniem coaching w pierwszej kolejności jest procesem. Ze Słownika Języka Polskiego można przytoczyć następującą definicję:
\begin{defn}
\emph{Proces} to przebieg następujących po sobie i powiązanych przyczynowo określonych zmian.
\end{defn}
Jednocześnie ten ciąg powiązanych ze sobą zmian ma być ograniczony czasowo. To założenie ma służyć dalszemu uniezależnieniu klienta
od coacha. Po zakończeniu procesu obie strony się rozstają, więc w najlepszym interesie klienta jest świadomie i jak najlepsze wykorzystanie
skończonej liczby sesji które może on odbyć.

\subsection{Coaching to partnerska współpraca}
Kolejną ważną koncepcją coachingu jest partnerski stosunek obu stron, w którym traktują one się wzajemnie w ten sam sposób. Dotyczy to zarówno
wymagań podstawowych, takich jak wzajemny szacunek czy zaangażowanie, ale również akceptacji i otwartości na inny punkt widzenia drugiej
osoby. Partnerska postawa powoduje, że w procesie coachingowym obie strony razem zmierzają do celu, ale jednocześnie obie zachowują
swoją niezależność. \\

Klasycznie współpraca ta jest budowana na fundamencie pytań, ale w swojej pracy coachowie wykorzystują również różnego rodzaju
narzędzia. Mają one na celu pobudzenie szukania kreatywnych rozwiązań przez klienta lub ułatwienie zaobserwowania
mu innego punktu widzenia danego zagadnienia.

\subsection{Coaching jest ukierunkowany na rozwój klienta}
Efektem procesu coachingowego powinna być trwała, pozytywna zmiana w życiu klienta. W przeciwieństwie do procesu terapeutycznego coaching
nie będzie miał na celu zdiagnozowania przyczyn zjawisk negatywnych, a skupi się na możliwych metodach ich przezwyciężenia. Jeszcze większej
uwagi wymaga ta część definicji kiedy w procesie uczestniczy również sponsor. Takim sponsorem może być na przykład pracodawca klienta.
To coach odpowiada za właściwe wytłumaczenie obowiązujących zasad klientowi i sponsorowi - w szczególności sponsor powinien mieć świadomość,
że w trakcie coachingu klient może podjąć decyzje niekorzystne z punktu widzenia sponsora.

\subsection{Coaching skupia się na teraźniejszości i przyszłości}
W celu ułatwienia klientowi przejęcia odpowiedzialności i ukierunkowania go na pozytywne rezultaty coaching koncentruje się na wydarzeniach
teraźniejszych i przyszłych. Coaching wychodzi z założenia, że na wydarzenia przeszłe nie ma się wpływu i powinno się je w pełni zaakceptować.
Oczywiście wyciąganie wniosków z przeszłych wydarzeń może okazać się cenną lekcją na przyszłość, ale nadmierne
rozpamiętywanie przeszłości będzie prawdopodobnie blokować klienta zamiast posuwać go naprzód.

\section{Porównanie z innymi formami rozwoju osobistego}
Przy definiowaniu \emph{coachingu} wartościowe jest umiejscowienie tego pojęcia względem innych form rozwoju osobistego.
W tabeli \ref{table:kategorie} został skategoryzowane najbardziej popularne formy rozwoju ze względu na swój dyrektywny
i niedyrektywny charakter oraz czas trwania procesu.

\begin{table}[!ht]
  \centering
  \caption*{Formy rozwoju osobistego}
  \def\arraystretch{1.5}
  \begin{tabular}{c|c|c|}
    \cline{2-3}
    & \emph{Dyrektywne} & \emph{Niedyrektywne} \\ \cline{1-3}
    \multicolumn{1}{|c|}{\multirow{3}{*}{\emph{Ograniczone czasowo}} } & Mentoring & Coaching \\
    \multicolumn{1}{|c|}{} & Nauka & Counseling \\
    \multicolumn{1}{|c|}{} & & Doradztwo \\ \cline{1-3}
    \multicolumn{1}{|c|}{\multirow{1}{*}{\emph{Nieograniczone czasowo}} } & Trening & Terapia \\ \cline{1-3}
  \end{tabular}
  \caption{Formy rozwoju osobistego skategoryzowane ze względu na czas trwania procesu oraz swój dyrektywny lub niedyrektywny charakter.}
  bel{table:kategorie}
\end{table}

\subsection{Czym różni się coaching od mentoringu?}

O ile w coachingu każdą decyzje lub wniosek klient podejmuje sam, to w mentoringu otrzymuje on sprawdzone, gotowe rozwiązania od mentora.
Mentoring często wykorzystywany jest w organizacjach, gdzie doświadczony pracownik jest mentorem dla osoby zaczynającej pracę na danym stanowisku.

Elementem procesu coachingowego będzie upewnienie się, że klient faktycznie chce zrealizować swój cel i że akceptuje jego konsekwencje.
Mentoring nie ma takiego holistycznego charakteru - skupia się na danym zagadnieniu i dostarcza wiedzę i umiejętności dla klienta.
Wspólnym mianownikiem jest ograniczenie czasowe nałożone na oba procesy.

\subsection{Czym różni się coaching od terapii?}
Główne różnice są dwie: \\
- Terapia nie musi być ograniczona czasowo i może mieć długotrwały charakter. \\
- W terapii klient nie przejmuje odpowiedzialności za proces - w procesie terapeutycznym terapeuta prowadzi klienta.

Podobieństwo między coachingiem, a terapią polega na ich niedyrektywnym charakterze. Terapeuta nie daje klientowi gotowych rozwiązań, które mogłyby
zostać bezpośrednio zastosowane w przypadku tej osoby.

\subsection{Czym różni się coaching od doradztwa?}
Doradztwo ma na celu dostarczenie klientowi wiedzy na temat jego obecnej sytuacji i możliwości które przed nim stoją. Doradca jako ekspert dokonuje
dokładnej analizy informacji otrzymanych od klienta i bazując na swoim doświadczeniu przedstawia klientowi propozycje, które uważa za najskuteczniejsze.
Z tego powodu doradca przejmuje część odpowiedzialności za wybory swojego klienta. Podobnie jak coaching proces doradczy jest ograniczony czasowo.

\subsection{Czym różni się coaching od nauki?}

Klasyczny proces uczenia polega na przekazaniu określonego zakresu wiedzy w ramach realizowanego programu, w określonym czasie. Uczeń ma za zadanie
przyswajać wiedzę teoretyczną oraz realizować ćwiczenia wyznaczane przez nauczyciela. Zazwyczaj pod koniec procesu nauki uczeń przystępuje do egzaminu
weryfikującego poziom jego wiedzy i umiejętności w danej dziedzinie. W porównaniu do coachingu w procesie nauki określone są bardzo sztywne wzorce
oraz mierzalne cele.

\subsection{Czym różni się coaching od counselingu?}

Podstawową różnicą pomiędzy coachingiem a counselingiem leży w postawionym celu. O ile coaching jest przede wszystkim nastawiony na realizację celów
klienta, to counseling ma za zadanie pomóc w rozwiązaniu pojawiających się problemów czy trudności. Jest on doradzany na przykład osobom zmagającym się
z wypaleniem zawodowym, problemami emocjonalnym lub w relacjach. Z tego powodu counseling pod niektórymi względami przypomina proces terapeutyczny.
Jednocześnie podobnie do coachingu w tym wypadku współpraca jest ograniczona czasowa.

\subsection{Czym różni się coaching od treningu?}

Trening jest długotrwałym procesem, w którym trener korzystając z swojej wiedzy i doświadczenia ustala realizowany plan treningowy. Poprzez wytrwały
trening oraz mierzenie osiąganych wyników wypracowywane są coraz lepsze rezultaty. Trening podobnie do mentoringu jest nastawiony na uzyskanie
jak najlepszych rezultatów, ale ma długotrwały charakter, w którym systematyczność i wytrwałość mają największe znaczenie.

\section{Porównanie z definicją ICF}
Podsumowanie tego rozdziału stanowi porównanie z definicją jednej z największych organizacji zrzeszających pasjonatów coachingu:
International Coach Federation (ICF) \cite{deficf}.

\begin{defn}
ICF definiuje coaching jako partnerstwo z klientem w prowokującym myślenie, kreatywnym procesie, który inspiruje ich do maksymalizacji ich
osobistego i zawodowego potencjału, co jest niezwykle istotne w dzisiejszym niepewnym i złożonym środowisku. (tłum. własne
\footnote{ICF defines coaching as partnering with clients in a thought-provoking and creative process that inspires them to maximize
their personal and professional potential, which is particularly important in today's uncertain and complex environment.})
\end{defn}

Bardzo czytelne podobieństwa pomiędzy tymi dwiema definicjami to bardzo wyraźne podkreślanie partnerskiego charakteru relacji klient-coach
oraz ukierunkowanie procesu na rozwój klienta. W definicji przedstawionej w tej pracy coach otrzymuje większą swobodę przy kreowaniu procesu, bez
potrzeby podkreślanie jego kreatywnej i prowokatywnej formy. Z drugiej strony zgodnie z tą definicją nacisk położony jest na skupienie się na
teraźniejszość i przyszłość, co nie jest dosłownie podkreślone w definicji ICF.

Istotną różnicą pomiędzy definicjami jest brak wymagania określenia ograniczonych ram czasowych dla procesu coachingowego. Również w innych materiałach
udostępnianych przez ICF dopuszczane są procesy trwające więcej niż wiele miesięcy, a sama istota określenia ram czasowych nie jest wyraźnie w tych materiałach
podkreślona. Zdaniem autora jest to bardzo istotne wymaganie i w sposób wartościowy wzbogaca ono definicję ICF.

Ostatnią z opisanych różnic jest opis dzisiejszego środowiska, mającego wpływ na istotność coachingu w dzisiejszym świecie. Zdaniem autora jest to część
nie przekazująca dodatkowych informacji czym coaching jest i przez to zbędna. Merytoryczne dyskusja na temat tego czy coaching jest w dzisiejszych czasach
niż w przeszłości mogłaby być ciekawym eksperymentem naukowym, ale nie ma ona wpływu na to czym coaching jest.
