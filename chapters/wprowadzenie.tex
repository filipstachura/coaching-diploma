\chapter{Wstęp}

% Sgormułowanie własnej, indywidualnej definicji coachingu (uszczegółowioną i uzasadnioną). W jaki sposób osobiście rozumiem czym jest Coaching oraz kim /jestem/staję się w roli coacha?

Niniejsza praca poświęcona jest doświadczeniom zebrane w trakcie studiów podyplomowych na kierunku \emph{Coaching profesjonalny} oraz wnioskom bazującym na tych doświadczeniach. Praza została podzielona na trzy części. We wprowadzeniu została sformułowana indywidualna definicja coachingu, która następnie została uzasadniona i porównana z innymi popularnymi definicjami. Kolejna część, stanowiącą trzon pracy, została podzielona na trzy podrozdziały:
\begin{itemize}
  \item studium przypadku pracy z klientem
  \item własne doświadczenia
  \item omówienie obszaru zastosowań procesu coachingowego
\end{itemize}
W części ostatniej przedstawiam swoją wizję dalszego osobistego rozwoju ze szczególnym uwzględnieniem umiejętności coachingowych. \\

Coaching jest forma wspomagania rozwoju, opartą o dialog pomiędzy coachem a klientem. Moim zdaniem ze względu na swój niedyrektywny charakter, a również często niemierzalność efektów procesu coachingowego forma ta doczekała się licznych definicji. Również w trakcie trwania studiów podyplomowych spotkałem się z wieloma różnymi definicjami, z których każda indywidualnie określa warunki tego procesu. Jednocześnie w zdecydowanej większości przypadków definicje te nie są wzajemnie sprzeczne, a uzupełniają się tworząc razem pełniejszy obraz i intuicję czym \emph{coaching} jest.

Pomimo wymienionych trudności w jednoznaczym zdefiniowaniu tej formy rozwoju istnieje szereg wymagań wspólnych dla wszystkich definicji. Moja indywidualna definicja coachingu jest następująca:
\begin{defn} \emph{Coaching} jest dobrowolnym i niedyrektywnym, ograniczonym czasowo procesem partnerskiej współpracy, ukierunkowanym na rozwój klienta poprzez skupienie na teraźniejszości i przyszłości.\end{defn}

Klasycznie proces coachingowy zbudowany jest na bazie pytań, ale w swojej pracy coachowie wykorzystują również inne narzędzia mające na celu samodzielne wypracowanie przez klienta nowych spostrzeżeń. Przy definiowaniu coachingu wartościowe jest umiejscowienie go względem innych form rozwoju osobistego. W tabeli \ref{table:kysymys} został skategoryzowane najbardziej popularne formy rozwoju ze względu na swój dyrektywny i niedyrektywny charakter oraz czas trwania procesu.

\begin{table}[!ht]
  \centering
  \caption*{Formy rozwoju osobistego}
\def\arraystretch{1.5}%  1 is the default, change whatever you need
\begin{tabular}{c|c|c|}

\cline{2-3}
& \emph{Dyrektywne} & \emph{Niedyrektywne} \\ \cline{1-3}
\multicolumn{1}{ |c|  }{\multirow{3}{*}{\emph{Ograniczone czasowo}} } & Mentoring & Coaching \\
\multicolumn{1}{ |c|  }{} & Nauka & Counselling    \\
\multicolumn{1}{ |c|  }{} & & Doradztwo    \\ \cline{1-3}
\multicolumn{1}{ |c|  }{\multirow{1}{*}{\emph{Nieograniczone czasowo}} } & Trening & Terapia \\ \cline{1-3}
\end{tabular}
\caption{Formy rozwoju osobistego skategoryzowane ze względu na czas trwania procesu oraz swój dyrektywny lub niedyrektywny charakter.}
\label{table:kysymys}
\end{table}

\TODO{porównanie z innymi formami i przejście po różnych wymiarach}
\TODO{jak moja definicja ma się do def. ICF: \url{http://icf.org.pl/pl150,czym-jest-coaching-a-czym-nie-jest.html} i ICC: \url{http://www.iccpoland.pl/pl/strefa_wiedzy/czym_jest_coaching}}
