\chapter{Wstęp}

% Sgormułowanie własnej, indywidualnej definicji coachingu (uszczegółowioną i uzasadnioną). W jaki sposób osobiście
% rozumiem czym jest Coaching oraz kim /jestem/staję się w roli coacha?

Niniejsza praca poświęcona jest doświadczeniom zebranym przez autora w trakcie studiów podyplomowych na kierunku
\emph{Coaching profesjonalny}. Omawiane są również obserwacje i wnioski bazujące na tych doświadczeniach.
Praza została podzielona na trzy części. We wprowadzeniu została sformułowana indywidualna definicja coachingu,
która następnie została uzasadniona i porównana z innymi popularnymi definicjami. Kolejna część, stanowiącą trzon pracy,
została podzielona na trzy podrozdziały, zawierające odpowiednio:
\begin{enumerate}
  \item Studium przypadku pracy z klientem, w którym dokładnie opisany jest jeden pełen proces coachingowy poprowadzony
      przez autora pracy w trakcie trwania studiów.
  \item Omówienie przez autora własnych doświadczenień związanych z coachingiem z perspektywy klienta, bazując na
      ćwiczeniach realizowanych z innymi studentami.
  \item Opinie na temat obszaru zastosowań procesu coachingowego oraz standardów etycznych.
\end{enumerate}
W ostatniej części pracy autor przedstawia swoją wizję dalszego rozwoju osobistego, ze szczególnym uwzględnieniem
obszaru umiejętności coachingowych. \\

\section{Indywidualna definicja coachingu}
Coaching jest forma wspomagania rozwoju, opartą o dialog pomiędzy coachem, to jest osobą prowadzącą proces, a klientem.
Ze względu na swój niedyrektywny charakter, a również często niemierzalność efektów procesu coachingowego forma ta doczekała się
licznych definicji. Również w trakcie trwania studiów podyplomowych studenci mogli spotkać się z wieloma różnymi definicjami,
zależnie od wykładowcy prowadzącego zajęcia. Każda z tych definicji na swój sposób określa warunki tego procesu.
Jednocześnie w zdecydowanej większości przypadków definicje te nie są względem siebie sprzeczne, a raczej uzupełniają się
wzajemnie, tworząc razem pełniejszy obraz i budując intuicję słuchaczy o tym czym \emph{coaching} jest.

Pomimo wymienionych trudności stojących na przeszkodzie jednoznacznego zdefiniowania tej formy rozwoju istnieje grupa wymagań wspólnych dla
wszystkich definicji. Będą to na przykład te wymagania, które mówią o tym, że coaching to proces lub że jest współpracą pomiędzy
coachem, a klientem. Indywidualna definicja autora tej pracy ma następującą treść:
\begin{defn}
  \emph{Coaching} jest dobrowolnym i niedyrektywnym, ograniczonym czasowo procesem partnerskiej współpracy,
  ukierunkowanym na rozwój klienta poprzez skupienie się na teraźniejszości i przyszłości.
  \label{definicja}
\end{defn}

W celu dokładnego wytłumaczenia przedstawionej definicji poszczególnie jej części zostały omówione niezależnie poniżej: \\

\textbf{Coaching jako proces dobrowolny} \\
Jednym z podstawowych założeń procesu coachingowego jest jego swobodna i nieobowiązkowa forma. Dotyczy to zarówno klienta, ale jest
również istotne z punktu widzenia coacha. Ponieważ obie strony mają świadomość swojego dobrowolnego pozwala to budować
zdrową relację opartą o niezależność wszystkich zaangażowanych osób. \\

\textbf{Coaching jako proces niedyrektywny} \\
Fundamentalnym założeniem jest powierzenie klientowi pełnej odpowiedzialności za podejmowane przez niego decyzje. W sposób naturany
warunek ten jest spełniony przez brak dyrektywnych rad i podpowiedzi od coacha. Ponieważ coach zadaje wyłącznie pytania klientowi,
a klient sam na podstawie pytań dochodzi do wniosków i rozwiązań, a ostatecznie swoich decyzji to naturalne jest branie odpowiedzialności
za te decyzję przez klienta. Zaakceptowanie tego wymagania powinno być warunkiem niezbędnym każdego procesu coachingowego.

Dodatkowym, bardzo ważnym czynnikiem jest również wysoka uwaga coacha na ustrzeganie się dyrektywności "\emph{nie wprost}". O takiej formie
dyrektywności mówi się wtedy, gdy coach przez zaniedbanie lub świadome działanie nakłania klienta do podjęcia pewnej decyzji bez uczciwego przeanalizowania
innych możliwości. \\

\textbf{Coaching jako proces ograniczony czasowo} \\
Zgodnie z tym stwierdzeniem coaching w pierwszej kolejności jest procesem. Ze Słownika Języka Polskiego można przytoczyć następującą definicję:
\begin{defn}
\emph{Proces} to przebieg następujących po sobie i powiązanych przyczynowo określonych zmian.
\end{defn}
Jednocześnie ten ciąg powiązanych ze sobą zmian ma być ograniczony czasowo. To założenie ma służyć dalszemu uniezależnieniu klienta
od coacha. Po zakończeniu procesu obie strony się rozstają, więc w najlepszym interesie klienta jest świadomie i jak najlepsze wykorzystanie
skończonej liczby sesji które może on odbyć. \\

\textbf{Coaching to partnerska współpraca} \\
Kolejną ważną koncepcją coachingu jest partnerski stosunek obu stron, w którym traktują one się wzajemnie w ten sam sposób. Dotyczy to zarówno
wymagań podstawowych, takich jak wzajemny szacunek czy zaangażowanie, ale również akceptacji i otwartości na inny punkt widzenia drugiej
osoby. Partnerska postawa powoduje, że w procesie coachingowym obie strony razem zmieżają do celu, ale jednocześnie obie zachowują
swoją niezależność. \\

Klasycznie współpraca ta jest budowana na fundamencie pytań, ale w swojej pracy coachowie wykorzystują również różnego rodzaju
narzędzia. Mają one na celu pobudzenie szukania kreatywnych rozwiązań przez klienta lub ułatwienie zaobserwowania
mu innego punktu widzenia danego zagadnienia. \\

\textbf{Coaching jest ukierunkowany na rozwój klienta} \\
Efektem procesu coachingowego powinna być trwała, pozytywna zmiana w życiu klienta. W przeciwieństwie do procesu terapeutycznego coaching
nie będzie miał na celu zdiagnozowania przyczyn zjawisk negatywnych, a skupii się na możliwych metodach ich przezwyciężenia. Jeszcze większej
uwagii wymaga ta część definicji kiedy w procesie uczestniczy również sponsor. Takim sponsorem może być na przykład pracodawca klienta.
To coach odpowiada za właściwe wytłumaczenie obowiązujących zasad klientowi i sponsorowi - w szczególności sponsor powinien mieć świadomość,
że w trakcie coachingu klient może podjąć decyzje niekorzystne z punktu widzenia sponsora. \\

\textbf{Coaching skupia się na teraźniejszości i przyszłości} \\
W celu ułatwienia klientowi przejęcia odpowiedzialności i ukierunkowania go na pozytywne rezultaty coaching koncentruje się na wydarzeniach
teraźniejszych i przyszłych. Coaching wychodzi z założenia, że na wydarzenia przeszłe nie ma się wpływu i powinno się je w pełni zaakceptować.
Oczywiście wyciąganie wniosków z przeszłych wydarzeń może okazać się cenną lekcją na przyszłość, ale nadmierne
rozpamiętywanie przeszłości będzie prawdopodobnie blokować klienta zamiast posuwać go naprzód.

\section{Porównanie z innymi formami rozwoju osobistego}
Przy definiowaniu coachingu wartościowe jest umiejscowienie go względem innych form rozwoju osobistego. W tabeli \ref{table:kategorie} został
skategoryzowane najbardziej popularne formy rozwoju ze względu na swój dyrektywny i niedyrektywny charakter oraz czas trwania procesu.

\begin{table}[!ht]
  \centering
  \caption*{Formy rozwoju osobistego}
  \def\arraystretch{1.5}%  1 is the default, change whatever you need
  \begin{tabular}{c|c|c|}
    \cline{2-3}
    & \emph{Dyrektywne} & \emph{Niedyrektywne} \\ \cline{1-3}
    \multicolumn{1}{ |c|  }{\multirow{3}{*}{\emph{Ograniczone czasowo}} } & Mentoring & Coaching \\
    \multicolumn{1}{ |c|  }{} & Nauka & Counselling    \\
    \multicolumn{1}{ |c|  }{} & & Doradztwo    \\ \cline{1-3}
    \multicolumn{1}{ |c|  }{\multirow{1}{*}{\emph{Nieograniczone czasowo}} } & Trening & Terapia \\ \cline{1-3}
  \end{tabular}
  \caption{Formy rozwoju osobistego skategoryzowane ze względu na czas trwania procesu oraz swój dyrektywny lub niedyrektywny charakter.}
  \label{table:kategorie}
\end{table}

\TODO{porównanie z innymi formami i przejście po różnych wymiarach}

\section{Porównanie z innymi popularnymi definicjami}
\TODO{jak moja definicja ma się do def. ICF: \url{http://icf.org.pl/pl150,czym-jest-coaching-a-czym-nie-jest.html} i ICC: \url{http://www.iccpoland.pl/pl/strefa_wiedzy/czym_jest_coaching}}
