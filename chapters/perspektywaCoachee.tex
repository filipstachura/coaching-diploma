
\section{Własne doświadczanie coachingu}

% Na podstawie sesji kiedy bylem osoba coachowana i cwiczen, ktore byly zwiazane z autocoachingiem.
% Jakie korzysci proces coachingowy przyniosl mi, na jakich poziomach piramidy Diltsa obserwuje zachodzace zmiany?
% Co ma dla mnie najwieksze znaczenie/ jest dla mnie najwieksza wartoscia z zaangazowania sie we wlasny proces coachingowy?

W tym rozdziale autor pracy omówi korzyści i zmiany pojawiające się w jego życiu z perspektywy osoby coachowanej.
Zostaną ona opisane wraz z informacją na którym poziomie piramidy Diltsa (rys. \ref{pirdil}) dana zmiana zachodzi.

\begin{figure}[htp]
\centering
\begin{tikzpicture}
\coordinate (A) at (-7,0) {};
\coordinate (B) at ( 7,0) {};
\coordinate (C) at (0,10) {};
\draw[name path=AC] (A) -- (C);
\draw[name path=BC] (B) -- (C);
\foreach \y/\A in {
0/Środowisko,
1/Zachowania,
2/Umiejętności,
3/Stany emocjonalne i samopoczucie,
4/Przekonania,
5/Wartości,
6/Tożsamość,
7/Misja ,
8/Duchowość} {
    \path[name path=horiz] (A|-0,\y) -- (B|-0,\y);
    \draw[name intersections={of=AC and horiz,by=P},
          name intersections={of=BC and horiz,by=Q}] (P) -- (Q)
          node[midway,above,align=center,text width=\dimexpr(12em-\y em)*3\relax] {\A};
}
\end{tikzpicture}
\caption{Poziomy neurologiczne opisane przez Roberta Diltsa i przedstawione w formie piramidy. \cite{dilts}}
\label{pirdil}
\end{figure}

\begin{enumerate}
  \item Zmiana na poziomie środowiska: Istotną zmianą na tym poziomie jest uczestnictwo w spotkaniach odbywających się
      w ramach realizowanych studiów podyplomowych z coachingu. Spotkania te odbywały się około raz w miesiącu, przez dwa dni.
      Uczestniczyły w nim osoby mocno zainteresowane tematyką rozwoju osobistego, a także wykładowcy przekazujący swoją
      wiedzę i doświadczenia. Pomimo faktu, że zajęcia te trwały jedynie kilkanaście godzin w miesiącu, to wprowadzenie
      tego typu zmiany do życia i otoczenie się osobami z takim doświadczeniem miało bardzo cenne skutki. Wielokrotnie
      autor pracy miał okazję spojrzeć na jakieś zagadnienie z zupełnie innej strony niż dotychczas, zdobyć nową umiejętność
      lub poszerzyć swoją wiedzę.

  \item Zmiany na poziomie zachowań: Zauważono zmiany zarówno w życiu osobistym jak i zawodowym. Nastąpiła poprawa sposobu
      prowadzenia spotkań i komunikacji w zespole. Autor zaczął używać lepiej sformułowanych pytań, ukierunkowanych na cel.
      Dodatkowo zaczął przywiązywać wagę do naturalnych predyspozycji komunikacyjnych swoich rozmówców i dostosowywać
      odpowiednio sam komunikat. Również w życiu osobistym coach poświęca większą uwagę drugiej osobie w trakcie rozmowy
      z nią. Co więcej dzięki pozyskaniu umiejętności coachingowych nastąpiło wprowadzenie nowego zachowania, tj. prowadzenie
      sesji autocoachingowych, odbywających się raz w miesiącu.

  \item Zmiana na poziomie umiejętności: W trakcie studiów pozyskane zostały umiejętności coachingowe, takie jak realizowanie
      ćwiczeń z wykorzystaniem narzędzi coachingowych, zadawanie właściwych pytań osobie coachowanej oraz większe skupienie na
      słuchaczu. Zmiany dotyczą również większej wrażliwości na emocje innych osób i właściwego rozpoznawania tych emocji.

  \item Zmiana na poziomie stanów emocjonalnych i samopoczucia: Dzięki zaangażowaniu się w proces rozwoju i realizację kolejnych
      jego etapów autor częściej odczuwa spokój i jest mniej zestresowany. Dodatkowo zauważył u siebie większa stabilność emocjonalną
      i odporność na wywieranie wpływu przez inne osoby. W ciągu roku poszerzona została świadomość dotycząca własnych zachowań, nawyków
      i ich wpływu na stany emocjonalne i poziomy energii w ciągu dnia.

  \item Zmiana na poziomie przekonań: Pierwsza zmiana dotyczy przekonań dotyczących samego coachingu. Autor pracy przed rozpoczęciem
      studiów miał różnego rodzaju przekonania na temat tego czym coaching jest i do jakich celów może służyć. W trakcie trwania
      tego roku te przekonania albo zostały zmodyfikowane albo całkowicie zmienione. Oprócz tego autor w ramach zajęć i wykonywanych
      na nich ćwiczeń wprowadził różne korzystne przekonania dot. sfery prywatnej i zawodowej, co następni zaowocowało np. wprowadzeniem nowych
      korzystnych nawyków.

  \item Zmiana na poziomie wartości: Realizowane ćwiczenia w ramach studiów zapoczątkowały dalsze procesy myślowe i doprowadziły do
      rozwiązania dylematów na poziomie wartości dot. życia zawodowego.

  \item Zmiana na poziomie misji: Zmiana dotyczy nie modyfikacji, ale rozszerzenia wcześniejszej wizji. Dzięki skupieniu się na tak fundamentalnych
      obszarach własnego życia autor zauważył możliwe cele długofalowe (perspektywa 20 letnia) oraz dołączył jeden z nich do swojego planu.
\end{enumerate}

Najbardziej istotną zmianą dla autora pracy jest aktywna i systematyczne zaangażowanie w proces swojego samorozwoju. Ta tematyka
od długiego czasu jest dla autora ważnym elementem jego życia, jednak dopiero systematyczna praca pozwoliła w tym obszarze osiągnąć
satysfakcję i poczucie spełnienia.
