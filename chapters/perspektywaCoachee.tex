
\section{Własne doświadczenie coachingu}

% Na podstawie sesji kiedy bylem osoba coachowana i cwiczen, ktore byly zwiazane z autocoachingiem.
% Jakie korzysci proces coachingowy przyniosl mi, na jakich poziomach piramidy Diltsa obserwuje zachodzace zmiany?
% Co ma dla mnie najwieksze znaczenie/ jest dla mnie najwieksza wartoscia z zaangazowania sie we wlasny proces coachingowy?

W tym rozdziale autor pracy omówi korzyści i zmiany pojawiające się w jego życiu z perspektywy osoby coachowanej.

\begin{figure}[htp]
\centering
\begin{tikzpicture}
\coordinate (A) at (-7,0) {};
\coordinate (B) at ( 7,0) {};
\coordinate (C) at (0,10) {};
\draw[name path=AC] (A) -- (C);
\draw[name path=BC] (B) -- (C);
\foreach \y/\A in {
0/Środowisko,
1/Zachowania,
2/Umiejętności,
3/Stany i samopoczucie,
4/Przekonania,
5/Wartości,
6/Tożsamość,
7/Misja ,
8/Duchowość} {
    \path[name path=horiz] (A|-0,\y) -- (B|-0,\y);
    \draw[name intersections={of=AC and horiz,by=P},
          name intersections={of=BC and horiz,by=Q}] (P) -- (Q)
          node[midway,above,align=center,text width=\dimexpr(12em-\y em)*3\relax] {\A};
}
\end{tikzpicture}
\caption{Poziomy neurologiczne opisane przez Roberta Diltsa i przedstawione w formie piramidy. \cite{dilts}}
\end{figure}

\begin{enumerate}
  \item Zmiana na poziomie zachowań: skuteczniejsza komunikacja w zespole. Wykorzystanie dobrych pytań. Dostosowanie
      metod komunikacji i zadań do predyspozycji rozmówcy.
  \item Zmiana na poziomie zachowań: skuteczniejsza komunikacja w relacjach osobistych, większa uwaga w procesie rozmowy
      z drugą osoba lub grupą.
  \item Zmiana na poziomie emocji: większa skuteczność i odporność na wywieranie wpływu przez innych, większa uwaga dot.
      własnych emocji i zachowań. Poszerzona świadomość odnośnie własnych procesów, zachowań, nawyków.
  \item Rozpoczęte sesje autocoachingowe. (Raz na miesiąc przez 6 miesięcy)
  \item Misja: zauważone możliwe cele długofalowe (perspektywa 20 lat)
  \item Tożsamość: Utwierdzenie się w zgodności moich decyzji życiowych względem moich potrzeb. (podejmowanie ryzyka,
      potrzeby tworzenia)
  \item Wartości: Uspokojone dylematy na poziomie wartości: swoboda, niezależność > stabilność, luksus
  \item Przekonania: zmiany na poziomie sesji wykonanych podczas zajęć (np. wprowadzenie nowych korzystnych nawyków)
  \item Umiejętności: umiejętności coachingowe - zadawania właściwych pytań, większe skupienie na słucheczu, rozpoznawanie
      stylu komunikacji interlokutora
  \item Środowisko: comiesięczne spotkania w gronie ludzi zainteresowanych samorozwojem z zestawem bardzo odmiennych umiejętności od moich
      i często z innym punktem widzenia
\end{enumerate}
