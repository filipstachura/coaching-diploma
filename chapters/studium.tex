
\chapter{Część zasadnicza}

\section{Studium przypadku}
% Opis jednego procesu coachingowego z zachowaniem poufności poprzez zapewnienie anonimowości osoby coachowanej (min. 5 sesji z jednym klientem.

W tej sekcji został opisany proces coachingowy realizowany w trakcie trwania studiów podyplomowych, prowadzony przez autora pracy.

\subsection{Początkowe wyzwania i cele klienta}
% punkt wyjścia - wyzwania i cele klienta

\begin{itemize}
  \item Pierwsze spotkanie miało na celu wybranie obszaru i celów do pracy nad. Klientka wybrała obszar zawodowy, gdzie po niedawno otrzymanym awansie stoją przed nią nowe wyzwania.
  \item Szczegółowe cele dotyczyły: poprawienia metod organizacji pracy członków koordynowanego zespołu, skuteczne stawianie wymagań, a także poprawienie metod przekazywania informacji zwrotnej.
  \item Dla każdego z celów klientka chciała określić mierzalne oznaki realizacji. Udało się je określić dla:..., nie udało się określić ich dla: ... .
\end{itemize}

\subsection{Proces coachingowy}
%) proces - nad czym pracowano na jakich poziomach zachodziły zmiany co stanowiło szególne wyzwanie w pracy akurat z tym klientem?
\begin{itemize}
  \item dylemat pomiędzy rolami - ekspert/inżynier VS lider projektu. Dylemat na poziomie wartości(?) - bezpośrednie tworzenie VS bycie mocą sprawczą. Klientka stwierdziła, że będzie bardziej
  skuteczna w drugiej roli, jednocześnie wykazywała duże pragnienie samorelizacji poprzez tworzenie.
  \item wykorzystane narzędzie Artura - figure it out. + diagram
  \item kilkukrotne zastosowanie GROW do mniejszych zadań pozwoliło klientce łatwiej przejmować odpowiedzialność w przyszłości. Można powiedzieć że wytwarza się naturalny wzorzec realizacji zadań/celów w oparciu o dostępne zasoby.
  \item wykorzystane narzędzie norman bennet - jakich zasobów potrzebuje (mentoring/coaching/nauka/doradztwo)
  \item wyzwanie dla mnie: - dylemat klienta jest uzasadniony, w wielu aspektach widać było chęć podąrzania w obydwu kierunkach ze przeświadczeniem że realizacja obu celów może doprowadzić do gorszych rezultatów. Staraliśmy się atakować problem od wielu stron i udało uzyskać rozsądny wynik, aczkolwiek ciężko stwierdzić że problem został całkowicie rozstrzygnięty (i pewnie to dobrze).
\end{itemize}

\begin{table}[!ht]
  \centering
  \caption*{Narzędzie coachingowe: ang. \emph{Figure it out} }
  \def\arraystretch{2}%  1 is the default, change whatever you need
  \begin{tabular}{|l|c|c|c|c|c|c|c|}
  \hline
  Zarząd & \includegraphics{img/s4} & \includegraphics{img/s4} & \includegraphics{img/s4}  & \includegraphics{img/s3} & \includegraphics{img/s2} & \includegraphics{img/s1} & \includegraphics{img/s4} \\ \hline
  Dyrektor & \includegraphics{img/s3} & \includegraphics{img/s3} & \includegraphics{img/s3} & \includegraphics{img/s2} & \includegraphics{img/s1} & \includegraphics{img/s3} & \includegraphics{img/s4} \\ \hline
  Kierownik & \includegraphics{img/s2} & \includegraphics{img/s2} & \includegraphics{img/s2} & \includegraphics{img/s1} & \includegraphics{img/s2} & \includegraphics{img/s2} & \includegraphics{img/s3}\\ \hline
  Specjalista & \includegraphics{img/s1} & \includegraphics{img/s1} & \includegraphics{img/s1} & \includegraphics{img/s1} & \includegraphics{img/s1} & \includegraphics{img/s1} & \includegraphics{img/s2}\\ \hline
  \end{tabular}
  \caption{\TODO{opis}}
  \label{table:figureitout}
\end{table}

\subsection{Uzyskane efekty i dalsze cele}
%) efekty - stan docelowoy - co zrealizowano, jakie dalsze postępowanie byłoby wskazane?

\begin{itemize}
  \item tak jak wyżej - GROW pozwolił wypracować skuteczny w klasycznych przypadkach wzorzec podejmowania decyzji
  \item klientka określiła kolejne cele związana ze znalezieniem mentora i samodzielnej nauki
\end{itemize}

\subsection{Retrospekcja z punktu widzenia coacha}
%) feedback - czego, jako coach w tym procesie dowiedziałem się o sobie/ swoich kompetencjach / postawie coacha - jakie dlasze zadania rozwojowe w roli coacha przede mna?}

\begin{itemize}
  \item Nikła wiedza na temat coachingu mimu kontaktu z wykształconą, zaradną osobą. Obawy wynikające z braku wiedzy
  \item Uzyskaliśmy dużo lepsze efekty niż przy pracy względem których istnieje wysokie zaangażowanie emocjonalne
  \item Wniosek klienta: sama też bym to potrafiła, ale dzięki temu naprawdę to robię
  \item Mój wniosek: Coach w pewnym sensie przypomina mentora ale to czego tak naprawdę uczy to \textbf{samodzielnego} rozwiązywania problemów.
\end{itemize}
