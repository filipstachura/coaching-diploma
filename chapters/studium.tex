
\chapter{Część zasadnicza}

\section{Studium przypadku}
% Opis jednego procesu coachingowego z zachowaniem poufności poprzez zapewnienie anonimowości osoby coachowanej (min. 5 sesji z jednym klientem.

W tej sekcji został opisany proces coachingowy realizowany w trakcie trwania studiów podyplomowych, prowadzony przez autora pracy. Klientką była
młoda kobieta na wczesnym etapie swojej kariery zawodowej. Jest to osoba konkretna, o wysokiej motywacji i nastawieniu na realizację celu.

Krótki czas przed rozpoczęciem procesu coachingowego klientka otrzymała awans, co sprawiło
że stanęła przed zupełnie nowymi wyzwaniami i z niektórymi współpracownikami musiała zmienić dotychczasową relację. W niektórych przypadkach oznaczało
to, że stała się przełożoną swoich dotychczasowych kolegów i koleżanek z pracy. Jej nowe obowiązki obejmowały między innymi definiowanie celów pośrednich,
harmonogramu projektów, dawanie informacji zwrotnej pracownikom, dbanie o poziom motywacji i atmosfery w zespole złożonym ze specjalistów. Warto podkreślić,
że klientka wcześniej sama obejmowała stanowisko eksperckie, na którym tego typu działania nie wchodziły w zakres jej obowiązków.

\subsection{Początkowe wyzwania i cele klienta}
% punkt wyjścia - wyzwania i cele klienta

Pierwsza odbyta sesja miała na celu potwierdzenie chęci obu stron do rozpoczęcia pracy coachingowej, następnie ustalenie kontraktu, aby potem wybrać obszar
i cele do realizacji w trakcie trwania okresu coachingu. Pierwszym wyzwaniem okazał się element edukacyjny. Pomimo, że klientka jest osobą dobrze
wykształconą, oczytaną i posiadającą wszechstronną wiedzę, jej wiedza na temat coachingu była na bardzo niskim poziomie. Coachowana\footnote{Proponowany
tłumaczenie angielskiego sformułowania \emph{coachee}, zgodnie z \url{http://sjp.pwn.pl/poradnia/haslo/coachee-mentee;10085.html}} wiedziała, że coaching
jest modną i zyskującą ciągle na popularności formą rozwoju osobistego. Na tych informacjach jej wiedza na temat coachingu się kończyła. Jednocześnie
klientka żywiła pewne obawy dotyczące samego procesu. Najważniejsza z nich dotyczyła zachowania niezależności w doborze tematów coachingu oraz prywatności
osobistej.

Autor pracy jako coach wyjaśnił klientce najważniejsze założenia coachingu oraz umiejscowił go względem innych form rozwoju osobistego, podobnie jak miało to miejsce
w rozdziale pierwszym. Klientka po wysłuchaniu i zrozumieniu tych informacji wyraziła gotowość do wejścia w proces coachingowy, decyzja była podjęta
z dużą gotowością na poznanie czegoś nowego oraz motywacją do wykorzystanie jak najlepiej pojawiającej się okazji rozwoju. Po tej deklaracji coach również
był gotowy do rozpoczęcia pracy, mając na uwadze posiadaną już przez klientkę wiedzę i jej wysoką motywację. W związku z tym obie strony przystąpiły
do ustalenia wspólnie kontraktu coachingowego.

Kontrakt zapewniał obu stronom pełną prywatność informacji omawianych w trakcie pracy coachingowej. Klientka otrzymała pełną kontrolę nad zakresem tematów
poruszanych na sesjach. Jest to całkowicie naturalne, że klient decyduje o poruszanych tematach, można więc zastanawiać się czy podkreślanie tego explicite
w kontrakcie jest istotne. W tym wypadku coach stwierdził, że ponieważ jest to obszar wątpliwy dla klientki to wyraźne podkreślenie tego ustalenia
będzie dla niej pomocne. Coach i coachowana umówili się również na dopilnowanie regularności swoich spotkań oraz dbanie o wysoki poziom energii na sesjach.

Ostatnim elementem pierwszej sesji było wyznaczenie tematyki i celów dla całego lub części procesu. Klientka sama przychodząc na sesję nie miała określonych
konkretnych zamiarów i celów. Coach wykorzystał narzędzie \emph{,,Koło życia''} co bardzo szybko pozwoliło stwierdzić coachowanej, że chce zająć się obszarem
rozwoju zawodowego. Decydujący wpływ na tę decyzję miał opisany wcześniej awans i nowe wyzwania z nim związane.

Po skutecznym określeniu obszaru, którego miał dotyczyć proces coachingowy, przystąpiono do ustalenia celów i efektów tego procesu. Po krótkiej przerwie
coach wykonał wraz z klientką około 20-minutowa sesję dodatkową, opartą o model \emph{GROW}. Wykorzystując ten model w kolejnych krokach omawia się:
\begin{enumerate}
  \item Cel (ang. \emph{goal}).
  \item Stan obecny - rzeczywistość (ang. \emph{reality}).
  \item Przeszkody i możliwości (ang. \emph{obstacles / options}).
  \item Wolę (ang. \emph{will}).
\end{enumerate}

W oparciu o ten model klientka zdefiniowała swoją obecną sytuację zawodową i zaobserwowała możliwości jakie stwarza jej obecna sytuacja. Mając świadomość
tego z czego może wybrać i ile ma zasobów podjęła decyzję o skupieniu się na trzech konkretnych umiejętnościach:
\begin{itemize}
  \item Umiejętności organizacji pracy członkom koordynowanego zespołu.
  \item Skutecznym stawianiu wymagań.
  \item Umiejętności właściwego i skutecznego przekazywania informacji zwrotnej.
\end{itemize}

Dla każdej z tych umiejętności klientka chciała określić mierzalne metody oceny, mające służyć sprawdzeniu stopnia poprawy w danym zakresie. Nie było
to zadanie łatwe dla klientki ponieważ wszystkie trzy umiejętności dotyczą tak zwanych kompetencji ,,miękkich''. Ostatecznie dla pierwszej umiejętności
klientka postanowiła wprowadzić w zespole system mierzenia opóźnień prac względem wyznaczonych wspólnie celów, a dla drugiej liczenie liczby przypadków
nieporozumień dotyczących wyznaczonych zadań. Największe problemy sprawiła ostatnia z umiejętności. W tym wypadku klientka postanowiła zapisywać własne
odczucia dotyczące przeprowadzanych rozmów i śledzić te notatki. Tego typu decyzja wynikła z oceny klientki, że w danym momencie jest w stanie z dużą
dozą pewności ocenić czy odbyta rozmowa była skuteczna i jednocześnie zachowująca inne stawiane przez nią wymagania (np. dbanie o atmosferę w zespole),
a jej obawy dotyczyły braku możliwości obserwowania procesu prac nad tą umiejętnością z perspektywy.

\subsection{Proces coachingowy}
%) proces - nad czym pracowano na jakich poziomach zachodziły zmiany co stanowiło szególne wyzwanie w pracy akurat z tym klientem?

Z perspektywy coacha zrealizowany proces coachingowy można podzielić na dwa etapy. W trakcie trwania pierwszego etapu w ramach sesji klientka
analizowała swoją obecną sytuację w życiu oraz rozwiązywała różnego rodzaju bieżące, pilne dla niej zagadnienia, w oparciu o metody coachingowe.
W trakcie trwania tego etapu można było zaobserwować wzrost samoświadomości dotyczącej własnych umiejętności, potrzeb, a także budowanie i przyswajanie
strategii skutecznego działania. Drugi etap był ukierunkowany na przyszłość i skupił się określeniu długofalowych potrzeb klientki i budowie rocznego
planu działania mającego być realizowanym samodzielnie po zakończeniu współpracy z coachem.

Podstawowym narzędziem pojawiającym się często w pierwszym etapie był model \emph{GROW}. Został on zastosowany kilkukrotnie do mniejszych zadań
którymi klientka chciała zająć się w ramach sesji. Standardowy schemat wyglądał w tym wypadku tak, że klientka przychodziła na sesję coachingową
z ustalonym zadaniem, na które chciała poświęcić czas w ramach coachingu. Ciekawą z punktu widzenia coacha obserwacją było wytwarzanie się, wraz z upływem
czasu, u klientki naturalnej strategii planowania działań. Z każdym kolejnym zastosowaniem modelu \emph{GROW} klientka szybciej potrafiła zastosować
go w pojawiających się sytuacjach, ale co ważniejsze - w sposób intuicyjny sama stosowała go indywidualnie poza sesjami coachingowymi. Można również liczyć,
że wzorce te zostały również przeniesione na inne obszary życia, niekoniecznie związane ze sferą zawodową. Ten wniosek wymagałby jednak dodatkowych
informacji ponieważ w ramach sesji poruszane były jedynie tematy zawodowe.

Bardzo ważnym, zdaniem coacha, tematem poruszonym w ramach jednej z pierwszych sesji był dylemat klientki dotyczący jej nowej roli.
Z jednej strony klientka wyrażała dużą mobilizację i chęć do realizacji w nowej roli. Z drugiej strony równocześnie wyrażała chęć realizacji
eksperckich zadań, co dawało jej w przeszłości bardzo dużą satysfakcję i pewność, że powierzone jej zadania są wykonywane na odpowiednim poziomie.
Wejście w rolę lidera sprawiło, że główny wkład w realizację celów nie był wymierny, a jedynie pośredni - to budziło w klientce niepokój.
Jednocześnie klientka wyraziła niepokój, że próba realizowania się w obu zadaniach może ją doprowadzić do porażki na obydwóch obszarach.

Najważniejszym, zdaniem coacha, ćwiczeniem dotyczącym tego dylematu był \emph{,,Diament kartezjański''}.
Coachowana zmierzyła się z dylematem poprzez odpowiedź na cztery pytania, gdzie za każdym razem zastanawiała się z czym związana jest zmiana
roli eksperta/inżyniera na rolę lidera:
\begin{enumerate}
\item Co się stanie, jeśli to zrobisz?
\item Co się stanie, jeśli tego nie zrobisz?
\item Co się nie stanie, jeśli to zrobisz?
\item Co się nie stanie, jeśli tego nie zrobisz?
\end{enumerate}
Dzięki wnioskom z tego ćwiczenia klientka podjęła decyzję o pozostaniu w nowej roli. Kluczowym czynnikiem było tutaj stwierdzenie, że posiada
ona predyspozycje do nowej roli i długofalowo będzie w niej bardziej skuteczna niż w roli inżyniera. Praca z tym dylematem była dla coacha
największym wyzwaniem w trakcie całego procesu coachingowego. Poruszony dylemat dotyczy obszarów wartości i tożsamości klienta. Interpretacja
decyzji klientki może być taka, że ważnym elementem jej życiowej misji jest tworzenie. Klientka oceniła więc, w której z ról będzie mogła skuteczniej
realizować swoją misję i na tej podstawie dokonała wyboru.

Z perspektywy coacha, autor pracy uważa, ze dylemat ten został na pewien czas załagodzony. Jako coach widzi tutaj własną możliwość wzrostu poprzez
skuteczniejszą możliwość pomocy klientowi w tak trudnych wyborach. Dodatkowo jako coach, autor jest zadowolony że decyzję i odpowiedzialność za nią
ponosi w pełni klient i w przyszłości zapewne powróci do tego dylematu, z większą wiedzą i doświadczeniem.

Kolejnym i ostatnim już ćwiczeniem omówionym w ramach tej pracy i dotyczącym pierwszego etapu jest narzędzie \emph{Figure it out} zaproponowane
w trakcie trwania studiów przez Artura Rzepackiego. W ramach tego ćwiczenia klient definiuje różnego rodzaju realizowane zadania w swojej
organizacji poprzez symbole, takie jak trójkąty, kwadraty, koła itd.. Następnie dla występujących w organizacji stanowisk przypisuje pewien
zbiór symboli, stałej wielkości, mający określić idealne zdaniem klienta wykorzystanie zasobów czasowych do realizacji obowiązków danej osoby.
Przykładowe przyporządkowanie symboli do stanowisk przedstawione jest na rysunku \ref{table:figureitout}.

\begin{table}[!ht]
  \centering
  \caption*{Narzędzie coachingowe: ang. \emph{Figure it out} }
  \def\arraystretch{2}%  1 is the default, change whatever you need
  \begin{tabular}{|l|c|c|c|c|c|c|c|}
  \hline
  Zarząd & \includegraphics{img/s4} & \includegraphics{img/s4} & \includegraphics{img/s4}  & \includegraphics{img/s3} & \includegraphics{img/s2} & \includegraphics{img/s1} & \includegraphics{img/s4} \\ \hline
  Dyrektor & \includegraphics{img/s3} & \includegraphics{img/s3} & \includegraphics{img/s3} & \includegraphics{img/s2} & \includegraphics{img/s1} & \includegraphics{img/s3} & \includegraphics{img/s4} \\ \hline
  Kierownik & \includegraphics{img/s2} & \includegraphics{img/s2} & \includegraphics{img/s2} & \includegraphics{img/s1} & \includegraphics{img/s2} & \includegraphics{img/s2} & \includegraphics{img/s3}\\ \hline
  Specjalista & \includegraphics{img/s1} & \includegraphics{img/s1} & \includegraphics{img/s1} & \includegraphics{img/s1} & \includegraphics{img/s1} & \includegraphics{img/s1} & \includegraphics{img/s2}\\ \hline
  \end{tabular}
  \caption{Przykładowe przyporządkowanie symboli realizowanych zadań do stanowisk w organizacji.}
  \label{table:figureitout}
\end{table}

Kolejnym krokiem przy realizacji tego ćwiczenia jest określenie zbioru symboli tej samej wielkości dla jego obecnej sytuacji.
Przy omawianym w tym rozdziale procesie coachingowym została również omówiona sytuacja członków zespołu prowadzonego przez klientkę.
Efektem ćwiczenia było przede wszystkim uświadomienie sobie przez klientkę, że nadal realizuje za dużo zadań związanych ze starą rolą ekspercką.
Drugim ciekawym wnioskiem była obserwacja, że w nowej roli jest również dla niej ważne uważne przyglądanie się i częściowa realizacja zadań
osób będących na jeszcze wyższych stanowiskach. \\

Drugi etap procesu był zauważalnie krótszy od pierwszego. W tym etapie klientka czuła się już znacznie swobodniej w nowej roli, ale również współpraca
z coachem stała się czymś bardziej naturalnym. Na tym etapie główny nacisk został położony na dalszy rozwój po zakończeniu procesu coachingowego
oraz ustalenie planu działania na najbliższy rok.

Do zaadresowania potrzeby dalszego rozwoju osobistego zostało wykorzystane narzędzie wprowadzone na zajęciach z Panem Maciejem Bennewiczem. W ramach tego
ćwiczenia klient analizuje cztery możliwe metody rozwoju: uczenie się, coaching, mentoring i doradztwo. Dla każdej z tych metod zastanawia się
co ważnego i cennego mógłby lub powinien zrobić z jej wykorzystaniem. Co może być interesujące, w ramach tego ćwiczenia coach otrzymał od klientki
niebezpośrednią informację zwrotną na temat korzyści wynikających jej zdaniem z zaangażowania się w proces coachingowy. Zostaną one omówione w
następnej sekcji.

W celu ustalenia planu działania na kolejny rok użyte zostały karty Dixit
\footnote{Gra karciana pobudzająca wyobraźnie - \url{https://pl.wikipedia.org/wiki/Dixit}}. Narzędzie to zostało zaprezentowane na zajęciach
prowadzonych przez Panię Katarzynę Ramirez-Cyzio. Klient w trakcie tego ćwiczenia wybiera kartę która reprezentuje dla niego sytuację po roku czasu,
a następnie układa pełny przebieg zdarzeń przy pomocy wybranych przez niego kart. Ćwiczenie można urozmaicać pytając na przykład o osoby, które powinny
pojawić się na drodze klienta. W dowolnym momencie klient poprawia układ kart na osi czasu, ale pod koniec ćwiczenia powinien zaakceptować finalny układ.

Na ostatniej sesji dzięki wykorzystaniu tego narzędzia został zbudowany plan działania na przyszłość. Ćwiczenie to pomogło również zakończyć
proces coachingowy w zrównoważony sposób i wycofać coacha z życia klientki.

\subsection{Uzyskane efekty i dalsze cele}
%) efekty - stan docelowoy - co zrealizowano, jakie dalsze postępowanie byłoby wskazane?

Z perspektywy całego procesu coachingowego określenie uzyskanych efektów i podsumowanie zaistniałych zmian zazwyczaj nie jest zadaniem łatwym.
Wynika to z faktu, że wiele z nich zachodzi w sposób długotrwały i niemierzalny. Przykładami takich zmian mogą być zmiany przekonań klienta lub
rozpoczynające proces rozwoju na jakimś polu - na przykład budowanie autorytetu wśród zespołu.

W celu zbudowania takiego podsumowania ważne staje się obserwowanie również zmian w postrzeganiu klienta i jego decyzji, a nie tylko jego zachowań
czy uzyskiwanych mierzalnych efektów. Ważnym elementem przy budowie tego typu podsumowania jest również uzyskanie opinii samego klienta na temat
przeprowadzonego procesu. W zależności czy klient postrzega poświęcony czas jako wartościowy lub bezwartościowy efektywność procesu będzie,
z bardzo dużym prawdopodobieństwem, różnić się diametralnie.

Do najbardziej istotnych efektów procesu należały:
\begin{itemize}
  \item Klientka nauczyła się samodzielnie i intuicyjnie stosować model \emph{GROW} do realizacji swoich celów zawodowych.
  \item W sposób regularny realizowane były realizowane cele wypracowywane w ramach sesji coachingowej. Dotyczyły one głównie wyznaczonych nach na
      pierwszej sesji obszarów: organizacji pracy, stawianie wymagań i dawania informacji zwrotnej.
  \item Jak klientka sama stwierdziła w trakcie jednego z ćwiczeń, największą korzyścią jej zdaniem było regularne poświęcanie czasu na zastanowienie
      się nad pojawiającymi się w jej życiu zagadnieniami. Zauważyła, że przy wysokim tempie życia wiele istotnych decyzji mogłoby zostać zmarginalizowanych,
      a dzięki spotkaniom w ramach sesji coachingowych poświęcała im właściwą uwagę. Prawdopodobnie cenne będzie tutaj dokładne przytoczenie słów klientki:
      \begin{quote}
      \centering
      \emph{,,Sama też bym to potrafiła, ale dzięki temu naprawdę to robię.''}
      \end{quote}
      Nawiązując do rozdziału pierwszego niniejszej pracy, w którym rozważana jest definicja coachingu, można zauważyć, że to jedno zdanie również
      może służyć za wymowną definicję procesu coachingowego.
\end{itemize}

Pod koniec procesu zostały wyznaczona następujące przyszłe cele:
\begin{itemize}
  \item Klientka określiła kolejne cele związana z rozwojem osobistym: pierwszym z nich jest znalezienie mentora, a drugi to pozyskiwanie dalszej wiedzy
      w ramach szkoleń i samodzielnej nauki.
  \item Ustalony został przeze klientkę roczny plan rozwoju zawodowego.
\end{itemize}

\subsection{Retrospekcja z punktu widzenia coacha}
% feedback - czego, jako coach w tym procesie dowiedziałem się o sobie/ swoich kompetencjach / postawie coacha
%          - jakie dlasze zadania rozwojowe w roli coacha przede mna?

Dla autora pracy był to pierwszy w pełni przeprowadzony proces coachingowy. To pierwsze doświadczenie doprowadziło do wielu cennych wniosków.

Pierwsza cenna wiedza płynie z samego początku procesu - pierwszego spotkania z klientem. Bardzo ważne jest dobrze przygotowanie się do tego spotkania,
ponieważ wcześniej coach nie ma żadnych informacji na temat stanu wiedzy swojego klienta. Coach powinien bez trudności wyjaśnić nie tylko czym jest coaching,
ale również czym charakteryzują się inne formy rozwoju i jakie są podobieństwa i różnice między nimi. Jest to szczególnie istotne w przypadku
klienta, dla którego są to pierwsze doświadczenia związane z coachingiem i może mieć złe wyobrażenie o tym jak wygląda taki proces.

Drugim wnioskiem, który może zostać wyciągnięty z tych doświadczeń jest skuteczność procesu coachingowego. W trakcie trwania procesu coach mógł zaobserwować
zmiany zachodzące w życiu klienta i usłyszeć jego zdanie na temat skuteczności wprowadzanych zmian. Dodatkowo coach zauważył, że zachodzące zmiany
mają głębszy charakter i są skuteczniejsze niż w przypadku pojedynczych sesji lub w przypadku zaangażowania emocjonalnego coacha w relację z osobą
coachowaną. Dzięki temu doświadczeniu coach nauczył się, że przeprowadzanie ćwiczeń na przykład z osobami bliskimi lub przyjaciółmi może być nieskuteczne
i jest zadaniem dużo trudniejszym niż ćwiczenie z osobami obcymi - względem których nie jest zaangażowanym się emocjonalnie.

Ostatnim wnioskiem jest podobieństwo roli coacha do roli mentora. Na początku studiów podyplomowych autor pracy miał poczucie, że są to dwie bardzo
dalekie od siebie formy rozwoju. Wraz ze zbieranym doświadczeniem zauważył jednak, że coach, podobnie jak mentor, również w pewnym sensie uczy swojego klienta.
W tym wypadku jednak nauka nie polega na przekazaniu konkretnej wiedzy, a metod analizy sytuacji z wielu perspektyw, kreatywnego podejścia lub
strategii myślenia skutecznych przy budowaniu strategii realizacji celów (takich jak \emph{GROW}). Po zakończeniu procesu coachingowego klient wyposażony
w te umiejętności jest gotowy do samodzielnego ich wykorzystywania w przyszłości.
