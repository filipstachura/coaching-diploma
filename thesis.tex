\documentclass{alk}

\usepackage{polski}
\usepackage{color}
\usepackage{tikz}
\usepackage{graphicx}
\usepackage{pbox}
\usepackage{amsmath}
\usepackage{mathtools}
\usepackage{algpseudocode}
\usepackage{amsthm}
\usepackage{caption}
\usepackage{subcaption}
\usepackage{minted}
\usepackage{epigraph}
\usepackage[nosingleletter]{impnattypo}
\usepackage{pictex}
\usepackage{rotating}
\usepackage{listings}
\usepackage{xcolor}

\newcommand\TODO[1]{\textcolor{red}{TODO: #1}}

\author{Filip Stachura}
\title{Studium przypadku oraz indywidualne doświadczenia i wnioski związane z procesem coachingowym}
\kierunek{Coaching profesjonalny}

% miesiąc i rok:
\date{Czerwiec 2015}

\usepackage{tikz}
\usetikzlibrary{shapes,snakes}

\usepackage{hyperref}
\hypersetup{
    colorlinks,
    citecolor=black,
    filecolor=black,
    linkcolor=black,
    urlcolor=black
}

\usepackage{fontspec}
\setmainfont{Arial}

\usepackage[left=2.5cm,top=2.5cm,right=2.5cm,bottom=2.5cm,bindingoffset=0cm]{geometry}

\usepackage{setspace}
\onehalfspacing

\usepackage{amsthm}
\theoremstyle{definition}
\newtheorem{defn}{Definicja}

\usepackage{multirow}

% koniec definicji

\begin{document}

\maketitle

\tableofcontents

\chapter*{Wstęp}
\addcontentsline{toc}{chapter}{Wstęp}

\TODO{Sgormułowanie własnej, indywidualnej definicji coachingu (uszczegółowioną i uzasadnioną). W jaki sposób osobiście rozumiem czym jest Coaching oraz kim /jestem/staję się w roli coacha?}


\chapter{Część zasadnicza}

\section{Studium przypadku}
% Opis jednego procesu coachingowego z zachowaniem poufności poprzez zapewnienie anonimowości osoby coachowanej (min. 5 sesji z jednym klientem.

W tej sekcji został opisany proces coachingowy realizowany w trakcie trwania studiów podyplomowych, prowadzony przez autora pracy.

\subsection{Początkowe wyzwania i cele klienta}
% punkt wyjścia - wyzwania i cele klienta

\begin{itemize}
  \item Pierwsze spotkanie miało na celu wybranie obszaru i celów do pracy nad. Klientka wybrała obszar zawodowy, gdzie po niedawno otrzymanym awansie stoją przed nią nowe wyzwania.
  \item Szczegółowe cele dotyczyły: poprawienia metod organizacji pracy członków koordynowanego zespołu, skuteczne stawianie wymagań, a także poprawienie metod przekazywania informacji zwrotnej.
  \item Dla każdego z celów klientka chciała określić mierzalne oznaki realizacji. Udało się je określić dla:..., nie udało się określić ich dla: ... .
\end{itemize}

\subsection{Proces coachingowy}
%) proces - nad czym pracowano na jakich poziomach zachodziły zmiany co stanowiło szególne wyzwanie w pracy akurat z tym klientem?
\begin{itemize}
  \item dylemat pomiędzy rolami - ekspert/inżynier VS lider projektu. Dylemat na poziomie wartości(?) - bezpośrednie tworzenie VS bycie mocą sprawczą. Klientka stwierdziła, że będzie bardziej
  skuteczna w drugiej roli, jednocześnie wykazywała duże pragnienie samorelizacji poprzez tworzenie.
  \item wykorzystane narzędzie Artura - figure it out. + diagram
  \item kilkukrotne zastosowanie GROW do mniejszych zadań pozwoliło klientce łatwiej przejmować odpowiedzialność w przyszłości. Można powiedzieć że wytwarza się naturalny wzorzec realizacji zadań/celów w oparciu o dostępne zasoby.
  \item wykorzystane narzędzie norman bennet - jakich zasobów potrzebuje (mentoring/coaching/nauka/doradztwo)
  \item wyzwanie dla mnie: - dylemat klienta jest uzasadniony, w wielu aspektach widać było chęć podąrzania w obydwu kierunkach ze przeświadczeniem że realizacja obu celów może doprowadzić do gorszych rezultatów. Staraliśmy się atakować problem od wielu stron i udało uzyskać rozsądny wynik, aczkolwiek ciężko stwierdzić że problem został całkowicie rozstrzygnięty (i pewnie to dobrze).
\end{itemize}

\begin{table}[!ht]
  \centering
  \caption*{Narzędzie coachingowe: ang. \emph{Figure it out} }
  \def\arraystretch{2}%  1 is the default, change whatever you need
  \begin{tabular}{|l|c|c|c|c|c|c|c|}
  \hline
  Zarząd & \includegraphics{img/s4} & \includegraphics{img/s4} & \includegraphics{img/s4}  & \includegraphics{img/s3} & \includegraphics{img/s2} & \includegraphics{img/s1} & \includegraphics{img/s4} \\ \hline
  Dyrektor & \includegraphics{img/s3} & \includegraphics{img/s3} & \includegraphics{img/s3} & \includegraphics{img/s2} & \includegraphics{img/s1} & \includegraphics{img/s3} & \includegraphics{img/s4} \\ \hline
  Kierownik & \includegraphics{img/s2} & \includegraphics{img/s2} & \includegraphics{img/s2} & \includegraphics{img/s1} & \includegraphics{img/s2} & \includegraphics{img/s2} & \includegraphics{img/s3}\\ \hline
  Specjalista & \includegraphics{img/s1} & \includegraphics{img/s1} & \includegraphics{img/s1} & \includegraphics{img/s1} & \includegraphics{img/s1} & \includegraphics{img/s1} & \includegraphics{img/s2}\\ \hline
  \end{tabular}
  \caption{\TODO{opis}}
  \label{table:figureitout}
\end{table}

\subsection{Uzyskane efekty i dalsze cele}
%) efekty - stan docelowoy - co zrealizowano, jakie dalsze postępowanie byłoby wskazane?

\begin{itemize}
  \item tak jak wyżej - GROW pozwolił wypracować skuteczny w klasycznych przypadkach wzorzec podejmowania decyzji
  \item klientka określiła kolejne cele związana ze znalezieniem mentora i samodzielnej nauki
\end{itemize}

\subsection{Retrospekcja z punktu widzenia coacha}
%) feedback - czego, jako coach w tym procesie dowiedziałem się o sobie/ swoich kompetencjach / postawie coacha - jakie dlasze zadania rozwojowe w roli coacha przede mna?}

\begin{itemize}
  \item Nikła wiedza na temat coachingu mimu kontaktu z wykształconą, zaradną osobą. Obawy wynikające z braku wiedzy
  \item Uzyskaliśmy dużo lepsze efekty niż przy pracy względem których istnieje wysokie zaangażowanie emocjonalne
  \item Wniosek klienta: sama też bym to potrafiła, ale dzięki temu naprawdę to robię
  \item Mój wniosek: Coach w pewnym sensie przypomina mentora ale to czego tak naprawdę uczy to \textbf{samodzielnego} rozwiązywania problemów.
\end{itemize}


\section{Własne doświadczanie coachingu}

% Na podstawie sesji kiedy bylem osoba coachowana i cwiczen, ktore byly zwiazane z autocoachingiem.
% Jakie korzysci proces coachingowy przyniosl mi, na jakich poziomach piramidy Diltsa obserwuje zachodzace zmiany?
% Co ma dla mnie najwieksze znaczenie/ jest dla mnie najwieksza wartoscia z zaangazowania sie we wlasny proces coachingowy?

W tym rozdziale autor pracy omówi korzyści i zmiany pojawiające się w jego życiu z perspektywy osoby coachowanej.
Zostaną ona opisane wraz z informacją na którym poziomie piramidy Diltsa (rys. \ref{pirdil}) dana zmiana zachodzi.

\begin{figure}[htp]
\centering
\begin{tikzpicture}
\coordinate (A) at (-7,0) {};
\coordinate (B) at ( 7,0) {};
\coordinate (C) at (0,10) {};
\draw[name path=AC] (A) -- (C);
\draw[name path=BC] (B) -- (C);
\foreach \y/\A in {
0/Środowisko,
1/Zachowania,
2/Umiejętności,
3/Stany emocjonalne i samopoczucie,
4/Przekonania,
5/Wartości,
6/Tożsamość,
7/Misja ,
8/Duchowość} {
    \path[name path=horiz] (A|-0,\y) -- (B|-0,\y);
    \draw[name intersections={of=AC and horiz,by=P},
          name intersections={of=BC and horiz,by=Q}] (P) -- (Q)
          node[midway,above,align=center,text width=\dimexpr(12em-\y em)*3\relax] {\A};
}
\end{tikzpicture}
\caption{Poziomy neurologiczne opisane przez Roberta Diltsa i przedstawione w formie piramidy. \cite{dilts}}
\label{pirdil}
\end{figure}

\begin{enumerate}
  \item Zmiana na poziomie środowiska: Istotną zmianą na tym poziomie jest uczestnictwo w spotkaniach odbywających się
      w ramach realizowanych studiów podyplomowych z coachingu. Spotkania te odbywały się około raz w miesiącu, przez dwa dni.
      Uczestniczyły w nim osoby mocno zainteresowane tematyką rozwoju osobistego, a także wykładowcy przekazujący swoją
      wiedzę i doświadczenia. Pomimo faktu, że zajęcia te trwały jedynie kilkanaście godzin w miesiącu, to wprowadzenie
      tego typu zmiany do życia i otoczenie się osobami z takim doświadczeniem miało bardzo cenne skutki. Wielokrotnie
      autor pracy miał okazję spojrzeć na jakieś zagadnienie z zupełnie innej strony niż dotychczas, zdobyć nową umiejętność
      lub poszerzyć swoją wiedzę.

  \item Zmiany na poziomie zachowań: Zauważono zmiany zarówno w życiu osobistym jak i zawodowym. Nastąpiła poprawa sposobu
      prowadzenia spotkań i komunikacji w zespole. Autor zaczął używać lepiej sformułowanych pytań, ukierunkowanych na cel.
      Dodatkowo zaczął przywiązywać wagę do naturalnych predyspozycji komunikacyjnych swoich rozmówców i dostosowywać
      odpowiednio sam komunikat. Również w życiu osobistym coach poświęca większą uwagę drugiej osobie w trakcie rozmowy
      z nią. Co więcej dzięki pozyskaniu umiejętności coachingowych nastąpiło wprowadzenie nowego zachowania, tj. prowadzenie
      sesji autocoachingowych, odbywających się raz w miesiącu.

  \item Zmiana na poziomie umiejętności: W trakcie studiów pozyskane zostały umiejętności coachingowe, takie jak realizowanie
      ćwiczeń z wykorzystaniem narzędzi coachingowych, zadawanie właściwych pytań osobie coachowanej oraz większe skupienie na
      słuchaczu. Zmiany dotyczą również większej wrażliwości na emocje innych osób i właściwego rozpoznawania tych emocji.

  \item Zmiana na poziomie stanów emocjonalnych i samopoczucia: Dzięki zaangażowaniu się w proces rozwoju i realizację kolejnych
      jego etapów autor częściej odczuwa spokój i jest mniej zestresowany. Dodatkowo zauważył u siebie większa stabilność emocjonalną
      i odporność na wywieranie wpływu przez inne osoby. W ciągu roku poszerzona została świadomość dotycząca własnych zachowań, nawyków
      i ich wpływu na stany emocjonalne i poziomy energii w ciągu dnia.

  \item Zmiana na poziomie przekonań: Pierwsza zmiana dotyczy przekonań dotyczących samego coachingu. Autor pracy przed rozpoczęciem
      studiów miał różnego rodzaju przekonania na temat tego czym coaching jest i do jakich celów może służyć. W trakcie trwania
      tego roku te przekonania albo zostały zmodyfikowane albo całkowicie zmienione. Oprócz tego autor w ramach zajęć i wykonywanych
      na nich ćwiczeń wprowadził różne korzystne przekonania dot. sfery prywatnej i zawodowej, co następni zaowocowało np. wprowadzeniem nowych
      korzystnych nawyków.

  \item Zmiana na poziomie wartości: Realizowane ćwiczenia w ramach studiów zapoczątkowały dalsze procesy myślowe i doprowadziły do
      rozwiązania dylematów na poziomie wartości dot. życia zawodowego.

  \item Zmiana na poziomie misji: Zmiana dotyczy nie modyfikacji, ale rozszerzenia wcześniejszej wizji. Dzięki skupieniu się na tak fundamentalnych
      obszarach własnego życia autor zauważył możliwe cele długofalowe (perspektywa 20 letnia) oraz dołączył jeden z nich do swojego planu.
\end{enumerate}

Najbardziej istotną zmianą dla autora pracy jest aktywna i systematyczne zaangażowanie w proces swojego samorozwoju. Ta tematyka
od długiego czasu jest dla autora ważnym elementem jego życia, jednak dopiero systematyczna praca pozwoliła w tym obszarze osiągnąć
satysfakcję i poczucie spełnienia.


\section{Zakres zastosowań coachingu}

% Jak teraz rozumiem zakres zastosowan coachingu?
% Ktore sposrod standardow etycznych ICF oraz Izby Coachingu uwazam za szczegolnie istotne z wlasnej praktyki zawodowej - uzasadnij z jakich powodow?
% Komu, kiedy i w jakich sytuacjach zaoferowalabym coaching?
% w jakich przypadkach przekazalbym klientowi kontakt do innego (jakiego?) specjalisy?

\subsection{W jakich sytuacjach można zastosować coaching?}

Zakres możliwych zastosowań coachingu jest niezwykle szeroki, ponieważ może on być stosowany bez względu na wiek, płeć, status społeczny etc.
W życiu każdej osoby wysoce prawdopodobne są sytuacje w których skorzystanie z procesu coachingowego może przynieść owocne rezultaty.

\subsection{Standardy etyczne w pracy coacha}
\begin{itemize}
  \item Z mojego doświadczenia punkt mówiący "będę zalecać klientowi korzystanie z usług innych specjalistów, kiedy uznam to za odpowiednie lub konieczne" jest szczególnie istotny.
      W obecnej sytuacji wśród osób prywatnych zainteresowanych rozwojem osobistym nadal nie ma dużej czytelności pomiędzy różnymi możliwymi metodami rozwoju.
      Osoby dla których bardziej odpowiednia mogłaby być np. terapia wysyłane są na coaching, głównie dlatego że jest on w tym momencie zyskującym na popularności zjawiskiem.
      Głęboko wierzę, że odpowiedzialność za edukowanie klientów o wadach i zaletach oraz formie różnych metod rozwoju osobistego leży w obowiązku każdego coacha.

      Sytuację tę komplikuje również gwałtowny wzrost zainteresowania coachingiem. Większość młodych ludzi zetknęła się z pojęciem \emph{coaching}, ale zaledwie niewielkie grono z nich
      ma prawidłową wiedzę na temat tego co to pojęcie oznacza. Wzrost zainteresowania coachingiem na terenie Polski jest przedstawiony na wykresie \ref{wykres}. Możemy na nim zaobserwować około czterokrotny wzrost
      ilości wyszukiwaniach w wyszukiwarce Google fraz związanych z coachingiem w latach 2007-2012, czyli zaledwie pięciu lat. Jednocześnie widać bardzo szybko malejące zainteresowanie doradztwem,
      które obecnie plasuje się na zbliżonym poziomie do coachingu. Wykres umożliwia również zaobserwowanie jak marginalny charakter względem coachingu i doradztwa mają takie formy samorozwoju
      jak mentoring czy counseling.

\begin{figure}[!ht]
  \centering
  \includegraphics[width=17cm]{img/popularnosc}
  \caption{\TODO{opis}}
  \label{wykres}
\end{figure}

  \item Drugim szczególnie istotnym punktem w moim przeświadczeniu jest dyskrecja w sprawach dotyczących klienta.
      Cytując za Kodeksem Etycznym Izby Coachingu: "6.2 Jeśli dobro Klienta wchodzi w konflikt z lojalnością zawodową, Coach pracuje przede wszystkim dla dobra Klienta i postępuje zgodnie z uzgodnionym kontraktem.". Widocznym wnioskiem z tego punktu jest jak niezwykle ważne staje się czytelne ustalanie kontraktu pomiędzy coachem a klientem. Szczególna uwaga powinna mieć miejsce jeśli w proces zaangażowany jest również zewnętrzny zleceniodawca.
\end{itemize}

\subsection{Grupa docelowa procesu coachignowego}

\begin{itemize}
  \item im bardziej samodzielnym tym lepiej, ale przynajmniej w obszarze celów procesu
  \item osobom o wysokiej motywacji
  \item znajdującym się w nowej dla nich sytuacji (którym prawdopodobnie towarzyszy wysoka motywacja pozytywna) lub
       ulegających wypaleniu lub frustracji (wysoka motywacja negatywna). W drugim przypadku należy również rozważyć
       mentoring/szkolenia (np. frustracja wynika z braków wiedzy i przez to nie są osiągane efekty) lub consuelling \TODO{pisownia?}.
  \item dla osób stojących przed dylematem i aktywnie szukających jego rozwiązania
\end{itemize}

Tak jak zostało to podkreślone w poprzedniej sekcji - bardzo istotnym aspektem pracy coacha jest wiedza o tym, kiedy klient powinien
zostac skierowany do innego specjalisty. Przy pomocy tabeli \ref{table:kategorie} można zidentyfikować następujące przypadki:
\begin{itemize}
\item[--] W przypadku osoby potrzebującej wiedzy i gotowych rozwiązań - mentoring lub szkolenia.
\item[--] W przypadku osoby o niskiej motywacji / wypaleniu - consuelling.
\item[--] W przypadku osób stojących przed decyzją, ale nie dylematem - tzn. potrzebujących wiedzy i oceny sytuacji - pomoc konsultant.
\item[--] W przypadku osób w stanach depresyjnych, nie widzących możliwych rozwiązań i niskiej motywacji do wprowadzania zmian w swoim życiu - terapia.
\end{itemize}

\chapter{Zakończenie}


\section{Wnioski}

\TODO{Jak moge wykorzystac umiejetnosci i doswiadczenia nabyte w trakcie studiow i kontynuowac swoj rozwoj w roli coacha? Jakie dzialania zamierzam podjac, kto bedzie stanowic moja docelowa nisze (z jakimi klientami i w jakich kontekstach mam zamiar pracowac?) Jaki zakres specjalizacji chcę rozwijac? Jakie dzialania praktyczne i kroki dalszego rozwoju (obszary doskonalenia) zamierzam podjac?}

\begin{thebibliography}{99}
\addcontentsline{toc}{chapter}{Bibliografia}

\bibitem{mitchell} Mitchell, T. (1997). \textit{Machine Learning}, McGraw Hill. ISBN 0-07-042807-7, p.2.

\bibitem{quinlan} Quinlan, J. R. (1993). \textit{Programs for Machine Learning}, Morgan Kaufmann Publishers.

\bibitem{quinlan2} Quinlan, J. R. (1996). \textit{Improved Use of Continuous Attributes in C4.5}, J. Artif. Intell. Res. vol 4, p. 77-90.

\bibitem{gama2} João Gama, (2010). \textit{Knowledge Discovery from Data Streams}, Chapman \& Hall/CRC, ISBN 978-1-4398-2611-9.

\bibitem{socketio} \textit{socket.io, Realtime application framework}, \url{https://github.com/Automattic/socket.io/wiki} (dostęp 12.2014)

\bibitem{wwwweka} \textit{Weka, Waikato Environment for Knowledge Analysis}, \url{https://weka.waikato.ac.nz} (dostęp 12.2014)

\bibitem{npmjava} \textit{java library, Bridge API to connect with existing Java APIs}, \url{https://www.npmjs.com/package/java} (dostęp 12.2014)

\bibitem{ibeacon} \textit{iBeacon for Developers}, \url{https://developer.apple.com/ibeacon/} (dostęp 12.2014)

\end{thebibliography}


\end{document}
