\documentclass{alk}

\usepackage{polski}
\usepackage{color}
\usepackage{tikz}
\usepackage{graphicx}
\usepackage{pbox}
\usepackage{amsmath}
\usepackage{mathtools}
\usepackage{algpseudocode}
\usepackage{amsthm}
\usepackage{caption}
\usepackage{subcaption}
\usepackage{minted}
\usepackage{epigraph}
\usepackage[nosingleletter]{impnattypo}
\usepackage{pictex}
\usepackage{rotating}
\usepackage{listings}
\usepackage{xcolor}

\newcommand\TODO[1]{\textcolor{red}{TODO: #1}}

\author{Filip Stachura}
\title{Wybór cech dla metod klasyfikacji działających na strumieniach danych w czasie rzeczywistym}
\kierunek{Coaching profesjonalny}

% miesiąc i rok:
\date{Czerwiec 2015}

\usepackage{hyperref}
\hypersetup{
    colorlinks,
    citecolor=black,
    filecolor=black,
    linkcolor=black,
    urlcolor=black
}

% koniec definicji

\begin{document}

\maketitle

%tu idzie streszczenie na strone poczatkowa
\begin{abstract}
  \TODO{zrobić}
\end{abstract}

\tableofcontents

\chapter*{Wstęp}
\addcontentsline{toc}{chapter}{Wstęp}

\TODO{Sgormułowanie własnej, indywidualnej definicji coachingu (uszczegółowioną i uzasadnioną). W jaki sposób osobiście rozumiem czym jest Coaching oraz kim /jestem/staję się w roli coacha?}


\chapter{Część zasadnicza}

\section{Studium przypadku}
% Opis jednego procesu coachingowego z zachowaniem poufności poprzez zapewnienie anonimowości osoby coachowanej (min. 5 sesji z jednym klientem.

W tej sekcji został opisany proces coachingowy realizowany w trakcie trwania studiów podyplomowych, prowadzony przez autora pracy.

\subsection{Początkowe wyzwania i cele klienta}
% punkt wyjścia - wyzwania i cele klienta

\subsection{Proces coachingowy}
%) proces - nad czym pracowano na jakich poziomach zachodziły zmiany co stanowiło szególne wyzwanie w pracy akurat z tym klientem?

\subsection{Uzyskane efekty i dalsze cele}
%) efekty - stan docelowoy - co zrealizowano, jakie dalsze postępowanie byłoby wskazane?

\subsection{Retrospekcja z punktu widzenia coacha}
%) feedback - czego, jako coach w tym procesie dowiedziałem się o sobie/ swoich kompetencjach / postawie coacha - jakie dlasze zadania rozwojowe w roli coacha przee mna?}


\section{Własne doświaczenie coachingu}

\TODO{na podstawie sesji kiedy bylem osoba coachowana i cwiczen, ktore byly zwiazane z autocoachingiem. Jakie korzysci proces coachingowy przyniosl mi, na jakich poziomach piramidy Diltsa obserwuje zachodzace zmiany? Co ma dla mnie najwieksze znaczenie/ jest dla mnie najwieksza wartoscia z zaangazowania sie we wlasny proces coachingowy?}

\section{Zakres zastosowań coachingu}

\TODO{Jak teraz rozumiem zakres zastosowan coachingu? Ktore sposrod standardow etycznych ICF oraz Izby Coachingu uwazam za szczegolnie istotne z wlasnej praktyki zawodowej - uzasadnij z jakich powodow? Komu, kiedy i w jakich sytuacjach zaoferowalabym coaching? w jakich przypadkach przekazalbym klientowi kontakt do innego (jakiego?) specjalisy?}

\chapter{Zakończenie}


\section{Wnioski}

\TODO{Jak moge wykorzystac umiejetnosci i doswiadczenia nabyte w trakcie studiow i kontynuowac swoj rozwoj w roli coacha? Jakie dzialania zamierzam podjac, kto bedzie stanowic moja docelowa nisze (z jakimi klientami i w jakich kontekstach mam zamiar pracowac?) Jaki zakres specjalizacji chcę rozwijac? Jakie dzialania praktyczne i kroki dalszego rozwoju (obszary doskonalenia) zamierzam podjac?}

\begin{thebibliography}{99}
\addcontentsline{toc}{chapter}{Bibliografia}

\bibitem{mitchell} Mitchell, T. (1997). \textit{Machine Learning}, McGraw Hill. ISBN 0-07-042807-7, p.2.

\bibitem{quinlan} Quinlan, J. R. (1993). \textit{Programs for Machine Learning}, Morgan Kaufmann Publishers.

\bibitem{quinlan2} Quinlan, J. R. (1996). \textit{Improved Use of Continuous Attributes in C4.5}, J. Artif. Intell. Res. vol 4, p. 77-90.

\bibitem{gama2} João Gama, (2010). \textit{Knowledge Discovery from Data Streams}, Chapman \& Hall/CRC, ISBN 978-1-4398-2611-9.

\bibitem{socketio} \textit{socket.io, Realtime application framework}, \url{https://github.com/Automattic/socket.io/wiki} (dostęp 12.2014)

\bibitem{wwwweka} \textit{Weka, Waikato Environment for Knowledge Analysis}, \url{https://weka.waikato.ac.nz} (dostęp 12.2014)

\bibitem{npmjava} \textit{java library, Bridge API to connect with existing Java APIs}, \url{https://www.npmjs.com/package/java} (dostęp 12.2014)

\bibitem{ibeacon} \textit{iBeacon for Developers}, \url{https://developer.apple.com/ibeacon/} (dostęp 12.2014)

\end{thebibliography}


\end{document}
